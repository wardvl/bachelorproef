%%=============================================================================
%% Inleiding
%%=============================================================================

\chapter{Inleiding}
\label{ch:inleiding}
% De inleiding moet de lezer net genoeg informatie verschaffen om het onderwerp te begrijpen en in te zien waarom de onderzoeksvraag de moeite waard is om te onderzoeken. In de inleiding ga je literatuurverwijzingen beperken, zodat de tekst vlot leesbaar blijft. Je kan de inleiding verder onderverdelen in secties als dit de tekst verduidelijkt. Zaken die aan bod kunnen komen in de inleiding~\autocite{Pollefliet2011}:
Als dochterbedrijf van de Xplore groep, ontwikkelt XTi open source applicaties in Java met de nieuwste technologieën. Bij XTi werken ze hoofdzakelijk aan green field projecten. Met andere woorden gaan ze aan de slag met een specifieke nood van de klant en ontwikkelen hier van nul een oplossing voor. Dit zorgt voor veel vrijheid in de keuze van gebruikte technologieën en gepaste architectuur voor het op te leveren project. Eens de applicatie wordt opgeleverd aan de klant, wordt het onderhoud van de applicatie overgelaten aan een externe partij. Hierbij is het van groot belang dat de opgeleverde applicaties goed onderhoudbaar zijn om interventies na de oplevering tot een minimum te beperken.

\section{Probleemstelling}
\label{sec:probleemstelling}
% Uit je probleemstelling moet duidelijk zijn dat je onderzoek een meerwaarde heeft voor een concrete doelgroep. De doelgroep moet goed gedefinieerd en afgelijnd zijn. Doelgroepen als ``bedrijven,'' ``KMO's,'' systeembeheerders, enz.~zijn nog te vaag. Als je een lijstje kan maken van de personen/organisaties die een meerwaarde zullen vinden in deze bachelorproef (dit is eigenlijk je steekproefkader), dan is dat een indicatie dat de doelgroep goed gedefinieerd is. Dit kan een enkel bedrijf zijn of zelfs één persoon (je co-promotor/opdrachtgever).
Voor projecten die een grote workload kennen of een groot aantal requests moeten verwerken wordt er tegenwoordig veelal gekozen voor een microservice architectuur. Dit is bij XTi niet anders. Keuze voor deze architectuur heeft tal van voordelen voor grootschalige oplossingen. De keerzijde van de medaille ligt echter in het feit dat, in applicaties met een microservice architectuur, het moeilijk kan zijn om te achterhalen waar er in de flow zich eventuele bottlenecks voordoen. Om dit fenomeen aan te pakken is XTi op zoek naar een oplossing die een antwoord biedt op dit probleem om zo hun klanten een betere totaaloplossing te kunnen bieden. Hierbij zijn zowel XTi, hun klanten en de externe partijen, die het onderhoudt van de projecten op zich nemen, gebaat bij een integratie voorzien met tracing mogelijkheden.

Voor het implementeren van tracing bestaan verschillende oplossingen. In dit onderzoek gaan we ons toespitsen op twee standaarden die het implementeren van tracing aanzienlijk makkelijker maken en een open standaard voorzien. Namelijk OpenTracing en OpenCensus. OpenTracing is een standaard die wordt ontwikkeld door \gls{CNCF}. Aan de andere kant hebben we OpenCensus. OpenCensus vindt zijn oorsprong bij Google, maar ondertussen helpt Microsoft mee aan de ontwikkeling van de standaard. Beide tools zijn opensource en zijn dus ook deels afhankelijk van de community voor de groei en succes ervan.

\section{Onderzoeksvraag}
\label{sec:onderzoeksvraag}
% Wees zo concreet mogelijk bij het formuleren van je onderzoeksvraag. Een onderzoeksvraag is trouwens iets waar nog niemand op dit moment een antwoord heeft (voor zover je kan nagaan). Het opzoeken van bestaande informatie (bv. ``welke tools bestaan er voor deze toepassing?'') is dus geen onderzoeksvraag. Je kan de onderzoeksvraag verder specifiëren in deelvragen. Bv.~als je onderzoek gaat over performantiemetingen, dan 
Door de opkomst van deze standaarden en hun opensource karakter zijn er talloze initiatieven die deze standaarden ondersteunen. Om te weten te komen welk van deze twee standaarden het meest matuur is en geschikt is om te gebruiken in toekomstige projecten van XTi, zal er in deze bachelorproef worden getracht een antwoord te formuleren op volgende onderzoeksvragen:

\begin{itemize}
	\item Welke van deze twee tools is het best geschikt om te implementeren binnen de stack van XTi?
	\item Wat zijn de voor- en nadelen van elke tool?
	\item Welke libraries zijn reeds voorzien van instrumentatie voor de standaard?
	\item Hoe groot en actief is de community voor de standaard?
	\item Wat is de aanpak om deze tools te implementeren/instrumenteren in een bestaande of nieuwe code base?
\end{itemize}

Dit zal verwezenlijkt worden op basis van een requirements analyse, onderzoek naar de community en de reeds geinstrumenteerde libraries en het opstellen van een proof of concept om zo een vergelijkende studie te maken tussen deze twee standaarden.

\section{Onderzoeksdoelstelling}
\label{sec:onderzoeksdoelstelling}
% Wat is het beoogde resultaat van je bachelorproef? Wat zijn de criteria voor succes? Beschrijf die zo concreet mogelijk.
Op het einde van deze bachelorproef moet het duidelijk zijn welk van deze twee standaarden het meest geschikt is om te gebruiken binnen de stack van XTi. Bovendien zal dit document als een leidraad moeten dienen om een van de gekozen standaarden te gaan implementeren in nieuwe projecten.

\section{Opzet van deze bachelorproef}
\label{sec:opzet-bachelorproef}

% Het is gebruikelijk aan het einde van de inleiding een overzicht te
% geven van de opbouw van de rest van de tekst. Deze sectie bevat al een aanzet
% die je kan aanvullen/aanpassen in functie van je eigen tekst.

De rest van deze bachelorproef is als volgt opgebouwd:

In Hoofdstuk~\ref{ch:stand-van-zaken} wordt een overzicht gegeven van de stand van zaken binnen het onderzoeksdomein, op basis van een literatuurstudie. In eerste instantie wordt er een uiteenzetting gegeven over wat de microservice architectuur is, wat de sterke punten er van zijn en wat de pitfalls zijn. In tweede instantie wordt er beschreven welke oplossingen er voor handen zijn om een antwoord te bieden op deze pitfalls.

In Hoofdstuk~\ref{ch:methodologie} wordt de methodologie toegelicht en worden de gebruikte onderzoekstechnieken besproken om een antwoord te kunnen formuleren op de onderzoeksvragen.

% TODO: Vul hier aan voor je eigen hoofstukken, één of twee zinnen per hoofdstuk

In Hoofdstuk~\ref{ch:conclusie}, tenslotte, wordt de conclusie gegeven en een antwoord geformuleerd op de onderzoeksvragen. Daarbij wordt ook een aanzet gegeven voor toekomstig onderzoek binnen dit domein.

