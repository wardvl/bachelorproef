%%=============================================================================
%% Voorwoord
%%=============================================================================

\chapter*{Woord vooraf}
\label{ch:voorwoord}

%% TODO:
%% Het voorwoord is het enige deel van de bachelorproef waar je vanuit je
%% eigen standpunt (``ik-vorm'') mag schrijven. Je kan hier bv. motiveren
%% waarom jij het onderwerp wil bespreken.
%% Vergeet ook niet te bedanken wie je geholpen/gesteund/... heeft
Voor u ligt de scriptie \"{}Vergelijkende studie tussen OpenTracing en OpenCensus: tracing \& metrics tussen microservices\"{}. Deze scriptie werd geschreven in functie van het behalen van mijn diploma Bachelor in de Toegepaste Informatica via afstandsonderwijs aan HoGent. Deze scriptie werd geschreven in opdracht van XTi.

In samenwerking met mijn promotor Sonia Vandermeersch en mijn co-promotor Bert Roex, heb ik onderzoek gepleegd naar de implementatie in Java van tracing tussen microservices. Ik was reeds voor dit onderzoek geïntrigeerd door het concept van microservices. Ik had er echter nog niet veel ervaring mee, zowel hands-on als theoretisch. Tijdens dit onderzoek heb ik veel bijgeleerd over de microservice architectuur en wat zijn voordelen maar ook gevaren zijn. Een leuk weetje: Tijdens dit onderzoek heb ik, volgens mijn geschiedenis op Youtube, voor een tijdsspanne van 12 uur en 24 minuten microservice en tracing gerelateerde talks bekeken. De ene al wat meer inspirerend dan de andere. Daarnaast heb ik natuurlijk ook heel wat artikels gelezen, maar hiervan heb ik geen statistieken voorhanden.

Graag neem ik hier ook de kans om iedereen te bedanken om mijn bachelorproef en studies tot een goed einde te brengen. Met name wens ik mijn promotor Sonia Vandermeersch te bedanken. Hoewel ik een man van de laaste minuut ben wist ze toch nog steeds de tijd op te brengen om goede feedback te verlenen. Ook wens ik Bert Roex te bedanken, zonder hem had dit document zelfs niet voor u gelegen. Ondanks het wegvallen van mijn initiële co-promotor die het onderwerp had aangereikt, heeft hij ervoor gekozen om mij toch dit onderwerp te laten uitwerken onder zijn vleugels.

Verder, maar niet allerminst, wens ik mijn vriendin Maaike te bedanken. Ze stond steeds paraat om mij tijdens de drukke periodes zo veel mogelijk te ontlasten, een luisterend oor te bieden, kortom mij bij te staan met raad en daad.

Ik wens u veel leesplezier.

Ward Vanlerberghe
